The Abstract of a thesis is a brief yet comprehensive summary of the research, providing an overview of its objectives, methodology, key findings, and conclusions. It serves as a standalone section that allows readers to quickly grasp the essence of the study without having to read the entire document. A well-structured abstract ensures that the research is accessible and effectively communicates its significance.

In terms of structure, the abstract should include a brief introduction to the research problem, followed by the main objectives or research questions. It should then summarize the methodology used, highlighting the key techniques and approaches. The abstract must also present the most important results and conclude with the main implications or contributions of the study. It should be written in a concise, objective, and impersonal style, avoiding citations, figures, or tables.

Regarding length, the abstract for a Master's thesis should not exceed 350 words, while for a Doctoral dissertation, it can be up to 500 words. It is important to ensure that the abstract is self-contained, meaning that it provides sufficient information for the reader to understand the research without additional context.