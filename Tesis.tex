% ==================================================================
% INICIA PREÁMBULO DEL DOCUMENTO
% ==================================================================

% -----------DECLARACIÓN DE TIPO DE DOCUMENTO A DISEÑAR-------------

\documentclass[12pt]{uaztesis}

% % --------------DECLARACIÓN DE PAQUETES A UTILIZAR------------------

\usepackage[spanish,mexico]{babel}
\usepackage{subfigure}
\usepackage{multirow,rotating}
\usepackage[hidelinks]{hyperref}
\usepackage{graphicx}
\usepackage{float}
\usepackage{fancyhdr}
\usepackage{listings}
\usepackage{amssymb,amsmath,ams fonts}
\usepackage{tabularx,colortbl,xspace,rotating,booktabs,longtable,multirow}
\usepackage{soulutf8}
\usepackage{multirow} % para las tablas
\usepackage{longtable} % para tablas largas
\usepackage{multirow, array} % para las tablas
\usepackage{float} % para usar [H]
\usepackage{color,curves}
\usepackage{chngcntr}
\usepackage{csquotes}
\usepackage{minted}
\renewcommand\listingscaption{C\'odigo}
\numberwithin{listing}{chapter}

% % --DECLARACIÓN DE RUTA DE CARPETA DONDE SE ENCUENTRAN LAS FIGURAS--
\graphicspath{{Imagenes/}}
\psfull

% % --------------DECLARACIÓN DEL ESTILO DE PÁGINA--------------------
\pagestyle{thesis}
\noappendixtables
\noappendixfigures

% % ===================================================================
% % INICIA EL CONTENIDO DEL DOCUMENTO
% % ===================================================================

\begin{document}

%\thesism % Tesis de maestria
\thesisd % Tesis de doctorado
%\thesis % Tesis licenciatura
%\degree{Ingeniero en Robótica y Mecatrónica} % Solo para licenciatura

% % Introducción en sílabas de palabras desconocidas por LaTeX
 \hyphenation{}

% % Declaración de número de páginas en números Romanos
 \clearpage\pagenumbering{roman}

% % ------------------INTRODUCCIÓN DE DATOS----------------------------

% % Título de la tesis, autor, y grado a recibir:
\title{Título de la tesis} 
\author{Estudiante}

% % Grado y nombre de asesores de tesis:
\advisortitle{Grado} \advisorname{Asesor}
\gradoasesor2{Grado} \nombreasesor2{Asesor}

\date{\the\year{}}

% % ---------GENERACIÓN DE PÁGINAS PRELIMINARES DEL DOCUMENTO----------

% % Genera página de presentación
 \maketitle

% % Genera páginas de resumen
\begin{resumen}
  Esto es el resumen
          
\end{resumen}

% % Genera página de dedicatoria (opcional)
\begin{dedicatoria}
  En esta sección se puede expresar su gratitud y reconocimiento a personas o entidades que han influido positivamente en su vida académica, profesional o personal, y que han brindado apoyo durante el proceso de investigación y redacción de la tesis. 

Es una parte personal y emotiva del documento, que permite al autor reconocer y honrar a quienes contribuyeron a su éxito y esfuerzo académico. Esta debe ser corta, en una sola página.          
\end{dedicatoria}

 % Genera página de agradecimientos (opcional)
\begin{agradecimientos}
  En esta sección se expresa su gratitud y reconocimiento a todas las personas e instituciones que han contribuido directa o indirectamente al desarrollo y culminación de su trabajo de investigación. Esta sección es una oportunidad para mostrar aprecio por el apoyo, la orientación y los recursos proporcionados durante el proceso de elaboración de la tesis.

La estructura es la siguiente:

\begin{enumerate}
    \item Al director de tesis
    \item A familiares y amigos
    \item A los compañeros de investigación
    \item Al Posgrado en Ingeniería para la Innovación Tecnológica y docentes
    \item A la Universidad Autónoma de Zacatecas
    \item Al CONAHCYT (en caso de ser becario)
\end{enumerate}

Un ejemplo de agradecimiento al CONAHCYT es: Al Sistema Nacional de Posgrados (SNP) del Consejo Nacional de Humanidades, Ciencia y Tecnología (CONAHCYT) por su apoyo económico a través de la convocatoria "Becas Nacionales (Tradicional) 20xx-20xx".        
\end{agradecimientos}

% % Genera páginas del contenido general y lista de figuras y tablas
\tableofcontents
\listoffigures                 
\listoftables                  

\begin{nomenclatura}
    \begin{description}
\item{\makebox[2.5cm][l]{$cd$}} Corriente directa.
\item{\makebox[2.5cm][l]{$ca$}} Corriente alterna.
\item{\makebox[2.5cm][l]{$NdFeB$}} Neudimio-Fierro-Boro.
\item{\makebox[2.5cm][l]{$FEM$}} Fuerza electromotriz.
\item{\makebox[2.5cm][l]{$FMM$}} Fuerza magnetomotriz.
\end{description}
\end{nomenclatura}

\begin{listofsymbols}
    \begin{description}
\item{\makebox[1.2cm][l]{$\mu_0$} \makebox[2.2cm][l] {$[Wb/A \cdot m]$}} Permeabilidad magnética del espacio libre.
\item{\makebox[1.2cm][l]{$\mu_{rFe}$} \makebox[2.2cm][l] { }} Permeabilidad magnética relativa del acero.
\item{\makebox[1.2cm][l]{$\rho$} \makebox[2.2cm][l] {$[kg/m^{3}]$}} Densidad de masa del aire.
\item{\makebox[1.2cm][l]{$\rho_{cu}$} \makebox[2.2cm][l] {$[kg/m^{3}]$}} Densidad de masa del cobre.
\item{\makebox[1.2cm][l]{$H$} \makebox[2.2cm][l] {$[A/m]$}} Intensidad de campo magnético.
\item{\makebox[1.2cm][l]{$\phi_r$} \makebox[2.2cm][l] {$[Wb]$}}  Flujo magnético remanente.
\item{\makebox[1.2cm][l]{$\eta$} \makebox[2.2cm][l] { }} Eficiencia.
\item{\makebox[1.2cm][l]{$\lambda$} \makebox[2.2cm][l] { }} Velocidad específica o TSR.
\item{\makebox[1.2cm][l]{$T$} \makebox[2.2cm][l] {$[N \cdot m]$}} Par o torque.
\item{\makebox[1.2cm][l]{$I$} \makebox[2.2cm][l] {$[A]$}} Corriente eléctrica.
\item{\makebox[1.2cm][l]{$R$} \makebox[2.2cm][l] {$[\Omega]$}} Resistencia eléctrica.
\end{description}
\end{listofsymbols}

\begin{glosario}
   \input{06_Glosario}
\end{glosario}

% %--------INCLUSIÓN DE LOS ARCHIVOS (CAPÍTULOS) DEL DOCUMENTO--------

% % Declaración de número de páginas en números Arábigos
 \clearpage\pagenumbering{arabic}

% incluye los captitulos de la tesis
\chapter{Introducción}

\section{Planteamiento del problema}

\section{Hipótesis}


\section{Justificación}


\section{Objetivos}

\subsection{Objetivo general}

\subsection{Objetivos particulares}

\section{Alcances y limitaciones}

\section{Metodología}
 








  






\chapter{Marco Teórico}

El marco teórico es una sección fundamental que proporciona la base conceptual y teórica sobre la cual se sustenta la investigación. En esta sección, se revisan y analizan las teorías, conceptos, estudios previos y trabajos relevantes que están directamente relacionados con el problema de investigación. El propósito del marco teórico es situar la investigación dentro del contexto académico y científico existente, y justificar el enfoque y los métodos seleccionados.

El marco teórico  debe proporcionar una base sólida y bien fundamentada que respalde la investigación. Al revisar la literatura existente, describir teorías y conceptos clave, y justificar el enfoque metodológico, este marco ayuda a situar la investigación dentro del contexto académico y científico adecuado.

Los componentes del Marco Teórico son:

\begin{enumerate}
    \item Introducción del Marco Teórico:
    \begin{enumerate}
        \item Breve introducción que explique el propósito del marco teórico y su relevancia para la investigación.
        \item Descripción del enfoque general y la estructura del marco teórico.
    \end{enumerate}
    \item Revisión de la Literatura:
    \begin{enumerate}
        \item Antecedentes Históricos: Contexto histórico del tema de estudio, incluyendo el desarrollo y la evolución de las teorías relevantes.
        \item Estudios Previos: Resumen y análisis de investigaciones previas relacionadas con el problema de investigación. Se destacan los hallazgos clave, metodologías utilizadas y brechas identificadas en el conocimiento.
        \item Estado del Arte: Exposición de las tecnologías, métodos y avances más recientes en el área de estudio. Esto puede incluir una revisión de las innovaciones tecnológicas, nuevas metodologías o enfoques teóricos.
    \end{enumerate}
    \item Teorías y Conceptos Relevantes:
    \begin{enumerate}
        \item Teorías Fundamentales: Descripción detallada de las teorías y modelos teóricos que forman la base de la investigación. Se debe explicar cómo estas teorías se aplican al problema de estudio.
        \item Conceptos Clave: Definición y explicación de los conceptos fundamentales utilizados en la investigación. Esto puede incluir términos técnicos, principios de ingeniería y conceptos científicos.
    \end{enumerate}
    \item Modelos y Enfoques Metodológicos:
    \begin{enumerate}
        \item Modelos Matemáticos y Computacionales: Presentación de los modelos teóricos y computacionales que serán utilizados o desarrollados en la investigación. Se deben explicar las ecuaciones y su relevancia.
        \item Metodologías de Investigación: Descripción de las metodologías y técnicas empleadas en estudios anteriores que son relevantes para el enfoque de la investigación actual.
    \end{enumerate}
    \item Relación del Marco Teórico con la Investigación:
    \begin{enumerate}
        \item Justificación del Enfoque: Explicación de cómo las teorías y conceptos revisados justifican y apoyan el enfoque metodológico elegido para la investigación.
        \item Integración de la Literatura: Síntesis de cómo los estudios previos, teorías y conceptos se integran para formar el marco conceptual de la investigación.
    \end{enumerate}
    \item Identificación de Brechas y Oportunidades:
    \begin{enumerate}
        \item Brechas en el Conocimiento: Identificación de las áreas que no han sido suficientemente exploradas o que presentan controversias en la literatura existente.
        \item Oportunidades de Investigación: Presentación de las oportunidades que estas brechas ofrecen para el desarrollo de nuevas investigaciones y contribuciones al campo de estudio.
    \end{enumerate}
\end{enumerate}
\chapter{Desarrollo}

\textbf{Cómo citar utilizando BibLaTeX}

BibLaTeX es una herramienta  para gestionar y generar referencias en documentos escritos con LaTeX. Ofrece una variedad de comandos para citar fuentes de manera adecuada según el contexto.

A continuación, se describe cómo usar los principales comandos para citar y se incluyen ejemplos prácticos.

\textbf{Comandos principales para citar}

\begin{enumerate}
    \item \textbackslash cite
    Este comando genera una cita compacta que usualmente incluye solo el autor y el año, o un número si se utiliza un estilo numérico.

    Uso típico: Se emplea en notas al pie o en listados donde no es necesario integrar la cita en el texto.
    
    Ejemplo:
    
    
    Según los resultados presentados en varios estudios \textbackslash cite\{smith2010data\}, el modelo es efectivo.
    
    Según los resultados presentados en varios estudios \cite{smith2010data}, el modelo es efectivo.
    
    \item \textbackslash parencite

    Genera una cita entre paréntesis que incluye el autor y el año o el identificador correspondiente al estilo elegido.
    Uso típico: Ideal para citas parentéticas en texto fluido.

    Ejemplo:

    El rendimiento de los sistemas GNSS ha sido evaluado (véase \textbackslash parencite\{johnson2015gps\}).


    El rendimiento de los sistemas GNSS ha sido evaluado (véase \parencite{johnson2015gps}).

    \item \textbackslash textcite

    Integra la referencia directamente en el flujo del texto, mencionando explícitamente el nombre del autor y el año.
    Uso típico: Perfecto para dar énfasis a los autores en el cuerpo del texto.

    Ejemplo:

    \textbackslash textcite\{doe2018antenna\} destacan que las antenas log-periódicas son altamente efectivas para aplicaciones de radar.

    \textcite{doe2018antenna} destacan que las antenas log-periódicas son altamente efectivas para aplicaciones de radar.

\end{enumerate}

\textbf{Ejemplo práctico de uso en un documento}

El desarrollo de antenas para aplicaciones de radar ha sido un tema de amplio interés. \textbackslash textcite\{smith2010data\} propusieron un diseño optimizado para frecuencias de microondas. Otros estudios, como los de \textbackslash parencite\{johnson2015gps\}, se han enfocado en la integración con sistemas GNSS.

Un análisis comparativo demuestra que \textbackslash cite\{doe2018antenna\} lograron mejores resultados al implementar estructuras log-periódicas.


El desarrollo de antenas para aplicaciones de radar ha sido un tema de amplio interés. \textcite{smith2010data} propusieron un diseño optimizado para frecuencias de microondas. Otros estudios, como los de \parencite{johnson2015gps}, se han enfocado en la integración con sistemas GNSS.

Un análisis comparativo demuestra que \cite{doe2018antenna} lograron mejores resultados al implementar estructuras log-periódicas.

\textbf{Citas con varias fuentes}

Cuando necesitas citar varias fuentes al mismo tiempo con BibLaTeX, puedes incluir múltiples claves de entrada en el mismo comando de citación, separadas por comas. Esto se aplica a los comandos como \textbackslash cite, \textbackslash parencite y \textbackslash textcite. A continuación, se muestra cómo hacerlo con ejemplos prácticos.

Ejemplo de citar varios autores al mismo tiempo

\begin{enumerate}
    \item Con \textbackslash cite

    Genera una cita compacta con todas las fuentes mencionadas.
    
    Diversos estudios han analizado el impacto de las antenas log-periódicas en aplicaciones de radar \textbackslash cite\{smith2010data, johnson2015gps, doe2018antenna\}.

    Diversos estudios han analizado el impacto de las antenas log-periódicas en aplicaciones de radar \cite{smith2010data, johnson2015gps, doe2018antenna}.

    \item Con \textbackslash parencite

    Incluye la lista de citas en un formato parentético.
    
    El rendimiento del sistema ha sido evaluado previamente (véase \textbackslash parencite\{smith2010data, johnson2015gps, doe2018antenna\}).

    El rendimiento del sistema ha sido evaluado previamente (véase \parencite{smith2010data, johnson2015gps, doe2018antenna}).

    \item Con \textbackslash textcite

    Integra varios autores en el texto.
    
    Trabajos recientes como los de \textbackslash textcite\{smith2010data\}, \textbackslash textcite\{johnson2015gps\} y \textbackslash textcite\{doe2018antenna\} han destacado diferentes aspectos del diseño y rendimiento de estas tecnologías.

    Trabajos recientes como los de \textcite{smith2010data}, \textcite{johnson2015gps} y \textcite{doe2018antenna} han destacado diferentes aspectos del diseño y rendimiento de estas tecnologías.
    
\end{enumerate}

\textbf{Resumen de recomendaciones para citar}

\begin{enumerate}
    \item \textbackslash cite: Úsalo para citas generales o en listados compactos.
    \item \textbackslash parencite: Utilízalo cuando necesites incluir la cita entre paréntesis.
    \item \textbackslash textcite: Prefiérelo para integrar la referencia de forma fluida en el texto.
\end{enumerate}

El archivo Referencias.bib es donde se almacenan las entradas bibliográficas.

Esta flexibilidad te permite adaptar tus citas al estilo de redacción que prefieras, manteniendo un formato académico correcto.
\chapter{Resultados}








\chapter{Conclusiones}



% % ------------------INTRODUCCIÓN DE REFERENCIAS---------------------

\nocite{*}
\chapter*{Referencias bibliográficas}     
\addcontentsline{toc}{chapter}{Referencias bibliográficas}
\printbibliography[heading=none]

% % ---------------------INTRODUCCIÓN DE APÉNDICES--------------------

\begin{apendices}
\chapter{Título}
\chapter{Título}


\end{apendices}


\end{document}