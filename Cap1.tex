\chapter{Introducción}

La introducción de una tesis es fundamental porque presenta el tema, los objetivos y la relevancia de la investigación.

Para ello se comienza con el Contexto y Antecedentes:

\begin{enumerate}
    \item Proporciona una visión general del tema de estudio.
    \item Explica por qué este tema es importante y relevante.
    \item Resalta los antecedentes históricos o teóricos necesarios para entender el problema de investigación.
\end{enumerate}

\section{Planteamiento del problema}

\begin{enumerate}
    \item Define claramente el problema específico que tu investigación abordará.
    \item Explica las brechas en el conocimiento existente que tu estudio pretende llenar.
    \item Justifica la necesidad de investigar este problema.
\end{enumerate}

\section{Hipótesis}

\begin{enumerate}
    \item En investigaciones cuantitativas, enuncia las hipótesis que serán probadas.
    \item Explica brevemente cómo se probarán estas hipótesis.
\end{enumerate}

\section{Justificación}

\begin{enumerate}
    \item Explica por qué tu investigación es importante.
    \item Destaca la relevancia académica, social, económica, o práctica del estudio.
    \item Menciona quiénes se beneficiarán de los hallazgos de tu investigación.
\end{enumerate}

\section{Objetivos}

\subsection{Objetivo general}

Presenta el propósito principal de tu estudio.

\subsection{Objetivos particulares}

Detalla los pasos específicos que tomarás para alcanzar el objetivo general.

\section{Alcances y limitaciones}

\begin{enumerate}
    \item Define los límites de tu investigación en términos de tiempo, espacio, y alcance.
    \item Especifica qué aspectos del tema no serán cubiertos y por qué.
    \item Discusión sobre las limitaciones del estudio y su impacto en los resultados.
    \item Definición del alcance del trabajo y lo que no se abordará en la investigación.

\end{enumerate}

\section{Metodología}

La metodología de la tesis sigue principios similares a los de otras disciplinas, pero tiende a ser más técnica y específica, ya que a menudo implica la creación de prototipos, experimentos, análisis de datos técnicos, simulaciones, y modelos matemáticos.

Componentes de la Metodología de la tesis:

\begin{enumerate}
    \item Enfoque de Investigación:
    \begin{enumerate}
        \item Teórico: Desarrollo de modelos matemáticos, simulaciones, o análisis teóricos.
        \item Experimental: Diseño y realización de experimentos para probar hipótesis o validar modelos.
        \item Desarrollo de Prototipos: Creación y evaluación de prototipos o sistemas.
        \item Mixto: Combinación de enfoques teóricos y experimentales.
    \end{enumerate}
    \item Diseño del Estudio:
    \begin{enumerate}
        \item Descripción del tipo de estudio (descriptivo, explicativo, experimental, de desarrollo de tecnología, etc.).
        \item Explicación de cómo se estructurará la investigación para abordar el problema planteado.
    \end{enumerate}
    \item Desarrollo de la Solución:
    \begin{enumerate}
        \item Análisis de Requisitos: Identificación y especificación de los requisitos del sistema o solución.
        \item Diseño del Sistema o Prototipo: Detalle del diseño conceptual y detallado, incluyendo diagramas, planos, o esquemas.
        \item Implementación: Descripción de los pasos para construir el prototipo o sistema, incluyendo materiales y herramientas utilizadas.
    \end{enumerate}
    \item Metodología Experimental:
    \begin{enumerate}
        \item Variables y Parámetros: Identificación de las variables independientes y dependientes, así como los parámetros de control.
        \item Procedimiento Experimental: Descripción detallada de cómo se llevarán a cabo los experimentos, incluyendo equipos y técnicas.
        \item Recolección de Datos: Métodos para recolectar datos, como sensores, instrumentos de medición, software de monitoreo, etc.
    \end{enumerate}
    \item Simulación y Modelado (si aplica):
    \begin{enumerate}
        \item Modelos Matemáticos: Descripción de los modelos matemáticos utilizados para representar el sistema o fenómeno.
        \item Herramientas de Simulación: Software y herramientas utilizadas para realizar simulaciones (MATLAB, ANSYS, etc.).
        \item Validación del Modelo: Métodos para validar los modelos, como comparación con datos experimentales.
    \end{enumerate}
    \item Análisis de Datos:
    \begin{enumerate}
        \item Técnicas y herramientas utilizadas para analizar los datos recolectados.
        \item Métodos estadísticos, gráficos y de visualización de datos.
        \item Evaluación del rendimiento del sistema o prototipo.
    \end{enumerate}
    \item Consideraciones Éticas y de Seguridad:
    \begin{enumerate}
        \item Descripción de las prácticas éticas y de seguridad aplicadas durante la investigación.
        \item Medidas para garantizar la seguridad de los experimentos y el cumplimiento de normas éticas.
    \end{enumerate}
\end{enumerate}

        

    
        
        

    
    
        

    
        
        

    








  





